% Options for packages loaded elsewhere
\PassOptionsToPackage{unicode}{hyperref}
\PassOptionsToPackage{hyphens}{url}
\PassOptionsToPackage{dvipsnames,svgnames,x11names}{xcolor}
%
\documentclass[
  letterpaper,
  DIV=11,
  numbers=noendperiod]{scrartcl}

\usepackage{amsmath,amssymb}
\usepackage{iftex}
\ifPDFTeX
  \usepackage[T1]{fontenc}
  \usepackage[utf8]{inputenc}
  \usepackage{textcomp} % provide euro and other symbols
\else % if luatex or xetex
  \usepackage{unicode-math}
  \defaultfontfeatures{Scale=MatchLowercase}
  \defaultfontfeatures[\rmfamily]{Ligatures=TeX,Scale=1}
\fi
\usepackage{lmodern}
\ifPDFTeX\else  
    % xetex/luatex font selection
\fi
% Use upquote if available, for straight quotes in verbatim environments
\IfFileExists{upquote.sty}{\usepackage{upquote}}{}
\IfFileExists{microtype.sty}{% use microtype if available
  \usepackage[]{microtype}
  \UseMicrotypeSet[protrusion]{basicmath} % disable protrusion for tt fonts
}{}
\makeatletter
\@ifundefined{KOMAClassName}{% if non-KOMA class
  \IfFileExists{parskip.sty}{%
    \usepackage{parskip}
  }{% else
    \setlength{\parindent}{0pt}
    \setlength{\parskip}{6pt plus 2pt minus 1pt}}
}{% if KOMA class
  \KOMAoptions{parskip=half}}
\makeatother
\usepackage{xcolor}
\setlength{\emergencystretch}{3em} % prevent overfull lines
\setcounter{secnumdepth}{-\maxdimen} % remove section numbering
% Make \paragraph and \subparagraph free-standing
\makeatletter
\ifx\paragraph\undefined\else
  \let\oldparagraph\paragraph
  \renewcommand{\paragraph}{
    \@ifstar
      \xxxParagraphStar
      \xxxParagraphNoStar
  }
  \newcommand{\xxxParagraphStar}[1]{\oldparagraph*{#1}\mbox{}}
  \newcommand{\xxxParagraphNoStar}[1]{\oldparagraph{#1}\mbox{}}
\fi
\ifx\subparagraph\undefined\else
  \let\oldsubparagraph\subparagraph
  \renewcommand{\subparagraph}{
    \@ifstar
      \xxxSubParagraphStar
      \xxxSubParagraphNoStar
  }
  \newcommand{\xxxSubParagraphStar}[1]{\oldsubparagraph*{#1}\mbox{}}
  \newcommand{\xxxSubParagraphNoStar}[1]{\oldsubparagraph{#1}\mbox{}}
\fi
\makeatother

\usepackage{color}
\usepackage{fancyvrb}
\newcommand{\VerbBar}{|}
\newcommand{\VERB}{\Verb[commandchars=\\\{\}]}
\DefineVerbatimEnvironment{Highlighting}{Verbatim}{commandchars=\\\{\}}
% Add ',fontsize=\small' for more characters per line
\usepackage{framed}
\definecolor{shadecolor}{RGB}{241,243,245}
\newenvironment{Shaded}{\begin{snugshade}}{\end{snugshade}}
\newcommand{\AlertTok}[1]{\textcolor[rgb]{0.68,0.00,0.00}{#1}}
\newcommand{\AnnotationTok}[1]{\textcolor[rgb]{0.37,0.37,0.37}{#1}}
\newcommand{\AttributeTok}[1]{\textcolor[rgb]{0.40,0.45,0.13}{#1}}
\newcommand{\BaseNTok}[1]{\textcolor[rgb]{0.68,0.00,0.00}{#1}}
\newcommand{\BuiltInTok}[1]{\textcolor[rgb]{0.00,0.23,0.31}{#1}}
\newcommand{\CharTok}[1]{\textcolor[rgb]{0.13,0.47,0.30}{#1}}
\newcommand{\CommentTok}[1]{\textcolor[rgb]{0.37,0.37,0.37}{#1}}
\newcommand{\CommentVarTok}[1]{\textcolor[rgb]{0.37,0.37,0.37}{\textit{#1}}}
\newcommand{\ConstantTok}[1]{\textcolor[rgb]{0.56,0.35,0.01}{#1}}
\newcommand{\ControlFlowTok}[1]{\textcolor[rgb]{0.00,0.23,0.31}{\textbf{#1}}}
\newcommand{\DataTypeTok}[1]{\textcolor[rgb]{0.68,0.00,0.00}{#1}}
\newcommand{\DecValTok}[1]{\textcolor[rgb]{0.68,0.00,0.00}{#1}}
\newcommand{\DocumentationTok}[1]{\textcolor[rgb]{0.37,0.37,0.37}{\textit{#1}}}
\newcommand{\ErrorTok}[1]{\textcolor[rgb]{0.68,0.00,0.00}{#1}}
\newcommand{\ExtensionTok}[1]{\textcolor[rgb]{0.00,0.23,0.31}{#1}}
\newcommand{\FloatTok}[1]{\textcolor[rgb]{0.68,0.00,0.00}{#1}}
\newcommand{\FunctionTok}[1]{\textcolor[rgb]{0.28,0.35,0.67}{#1}}
\newcommand{\ImportTok}[1]{\textcolor[rgb]{0.00,0.46,0.62}{#1}}
\newcommand{\InformationTok}[1]{\textcolor[rgb]{0.37,0.37,0.37}{#1}}
\newcommand{\KeywordTok}[1]{\textcolor[rgb]{0.00,0.23,0.31}{\textbf{#1}}}
\newcommand{\NormalTok}[1]{\textcolor[rgb]{0.00,0.23,0.31}{#1}}
\newcommand{\OperatorTok}[1]{\textcolor[rgb]{0.37,0.37,0.37}{#1}}
\newcommand{\OtherTok}[1]{\textcolor[rgb]{0.00,0.23,0.31}{#1}}
\newcommand{\PreprocessorTok}[1]{\textcolor[rgb]{0.68,0.00,0.00}{#1}}
\newcommand{\RegionMarkerTok}[1]{\textcolor[rgb]{0.00,0.23,0.31}{#1}}
\newcommand{\SpecialCharTok}[1]{\textcolor[rgb]{0.37,0.37,0.37}{#1}}
\newcommand{\SpecialStringTok}[1]{\textcolor[rgb]{0.13,0.47,0.30}{#1}}
\newcommand{\StringTok}[1]{\textcolor[rgb]{0.13,0.47,0.30}{#1}}
\newcommand{\VariableTok}[1]{\textcolor[rgb]{0.07,0.07,0.07}{#1}}
\newcommand{\VerbatimStringTok}[1]{\textcolor[rgb]{0.13,0.47,0.30}{#1}}
\newcommand{\WarningTok}[1]{\textcolor[rgb]{0.37,0.37,0.37}{\textit{#1}}}

\providecommand{\tightlist}{%
  \setlength{\itemsep}{0pt}\setlength{\parskip}{0pt}}\usepackage{longtable,booktabs,array}
\usepackage{calc} % for calculating minipage widths
% Correct order of tables after \paragraph or \subparagraph
\usepackage{etoolbox}
\makeatletter
\patchcmd\longtable{\par}{\if@noskipsec\mbox{}\fi\par}{}{}
\makeatother
% Allow footnotes in longtable head/foot
\IfFileExists{footnotehyper.sty}{\usepackage{footnotehyper}}{\usepackage{footnote}}
\makesavenoteenv{longtable}
\usepackage{graphicx}
\makeatletter
\newsavebox\pandoc@box
\newcommand*\pandocbounded[1]{% scales image to fit in text height/width
  \sbox\pandoc@box{#1}%
  \Gscale@div\@tempa{\textheight}{\dimexpr\ht\pandoc@box+\dp\pandoc@box\relax}%
  \Gscale@div\@tempb{\linewidth}{\wd\pandoc@box}%
  \ifdim\@tempb\p@<\@tempa\p@\let\@tempa\@tempb\fi% select the smaller of both
  \ifdim\@tempa\p@<\p@\scalebox{\@tempa}{\usebox\pandoc@box}%
  \else\usebox{\pandoc@box}%
  \fi%
}
% Set default figure placement to htbp
\def\fps@figure{htbp}
\makeatother

% load packages
\usepackage{geometry}
\usepackage{xcolor}
\usepackage{eso-pic}
\usepackage{fancyhdr}
\usepackage{sectsty}
\usepackage{fontspec}
\usepackage{titlesec}

%% Set page size with a wider right margin
\geometry{a4paper, total={170mm,257mm}, left=20mm, top=20mm, bottom=20mm, right=50mm}

%% Let's define some colours
\definecolor{light}{HTML}{E6E6FA}
\definecolor{highlight}{HTML}{800080}
\definecolor{dark}{HTML}{330033}

%% Let's add the border on the right hand side 
\AddToShipoutPicture{% 
    \AtPageLowerLeft{% 
        \put(\LenToUnit{\dimexpr\paperwidth-3cm},0){% 
            \color{light}\rule{3cm}{\LenToUnit\paperheight}%
          }%
     }%
     % logo
    \AtPageLowerLeft{% start the bar at the bottom right of the page
        \put(\LenToUnit{\dimexpr\paperwidth-2.25cm},27.2cm){% move it to the top right
            \color{light}\includegraphics[width=1.5cm]{_extensions/nrennie/PrettyPDF/logo.png}
          }%
     }%
}

%% Style the page number
\fancypagestyle{mystyle}{
  \fancyhf{}
  \renewcommand\headrulewidth{0pt}
  \fancyfoot[R]{\thepage}
  \fancyfootoffset{3.5cm}
}
\setlength{\footskip}{20pt}

%% style the chapter/section fonts
\chapterfont{\color{dark}\fontsize{20}{16.8}\selectfont}
\sectionfont{\color{dark}\fontsize{20}{16.8}\selectfont}
\subsectionfont{\color{dark}\fontsize{14}{16.8}\selectfont}
\titleformat{\subsection}
  {\sffamily\Large\bfseries}{\thesection}{1em}{}[{\titlerule[0.8pt]}]
  
% left align title
\makeatletter
\renewcommand{\maketitle}{\bgroup\setlength{\parindent}{0pt}
\begin{flushleft}
  {\sffamily\huge\textbf{\MakeUppercase{\@title}}} \vspace{0.3cm} \newline
  {\Large {\@subtitle}} \newline
  \@author
\end{flushleft}\egroup
}
\makeatother

%% Use some custom fonts
\setsansfont{Ubuntu}[
    Path=_extensions/nrennie/PrettyPDF/Ubuntu/,
    Scale=0.9,
    Extension = .ttf,
    UprightFont=*-Regular,
    BoldFont=*-Bold,
    ItalicFont=*-Italic,
    ]

\setmainfont{Ubuntu}[
    Path=_extensions/nrennie/PrettyPDF/Ubuntu/,
    Scale=0.9,
    Extension = .ttf,
    UprightFont=*-Regular,
    BoldFont=*-Bold,
    ItalicFont=*-Italic,
    ]
\KOMAoption{captions}{tableheading}
\makeatletter
\@ifpackageloaded{caption}{}{\usepackage{caption}}
\AtBeginDocument{%
\ifdefined\contentsname
  \renewcommand*\contentsname{Table of contents}
\else
  \newcommand\contentsname{Table of contents}
\fi
\ifdefined\listfigurename
  \renewcommand*\listfigurename{List of Figures}
\else
  \newcommand\listfigurename{List of Figures}
\fi
\ifdefined\listtablename
  \renewcommand*\listtablename{List of Tables}
\else
  \newcommand\listtablename{List of Tables}
\fi
\ifdefined\figurename
  \renewcommand*\figurename{Figure}
\else
  \newcommand\figurename{Figure}
\fi
\ifdefined\tablename
  \renewcommand*\tablename{Table}
\else
  \newcommand\tablename{Table}
\fi
}
\@ifpackageloaded{float}{}{\usepackage{float}}
\floatstyle{ruled}
\@ifundefined{c@chapter}{\newfloat{codelisting}{h}{lop}}{\newfloat{codelisting}{h}{lop}[chapter]}
\floatname{codelisting}{Listing}
\newcommand*\listoflistings{\listof{codelisting}{List of Listings}}
\makeatother
\makeatletter
\makeatother
\makeatletter
\@ifpackageloaded{caption}{}{\usepackage{caption}}
\@ifpackageloaded{subcaption}{}{\usepackage{subcaption}}
\makeatother
\makeatletter
\@ifpackageloaded{tcolorbox}{}{\usepackage[skins,breakable]{tcolorbox}}
\makeatother
\makeatletter
\@ifundefined{shadecolor}{\definecolor{shadecolor}{rgb}{.97, .97, .97}}{}
\makeatother
\makeatletter
\@ifundefined{codebgcolor}{\definecolor{codebgcolor}{named}{light}}{}
\makeatother
\makeatletter
\ifdefined\Shaded\renewenvironment{Shaded}{\begin{tcolorbox}[frame hidden, boxrule=0pt, enhanced, colback={codebgcolor}, sharp corners, breakable]}{\end{tcolorbox}}\fi
\makeatother

\usepackage{bookmark}

\IfFileExists{xurl.sty}{\usepackage{xurl}}{} % add URL line breaks if available
\urlstyle{same} % disable monospaced font for URLs
\hypersetup{
  pdftitle={GitHub Basics},
  pdfauthor={Md. Jubayer Hossain},
  colorlinks=true,
  linkcolor={highlight},
  filecolor={Maroon},
  citecolor={Blue},
  urlcolor={highlight},
  pdfcreator={LaTeX via pandoc}}


\title{GitHub Basics}
\usepackage{etoolbox}
\makeatletter
\providecommand{\subtitle}[1]{% add subtitle to \maketitle
  \apptocmd{\@title}{\par {\large #1 \par}}{}{}
}
\makeatother
\subtitle{A guide to get you started on GitHub}
\author{Md. Jubayer Hossain}
\date{}

\begin{document}
\maketitle

\pagestyle{mystyle}


\subsection{1. Terminology}\label{terminology}

\subsubsection{1.1 Fork}\label{fork}

\begin{itemize}
\tightlist
\item
  \textbf{What it means}: A fork is your own personal copy of someone
  else's repository. You can freely make changes to your forked copy
  without affecting the original repository.
\item
  \textbf{When to fork}: You fork a repository if you want to develop
  your own version or if you are \emph{not} a direct collaborator on the
  main repository but want to propose changes via Pull Requests.
\end{itemize}

\subsubsection{1.2 Clone}\label{clone}

\begin{itemize}
\tightlist
\item
  \textbf{What it means}: Cloning a repository means creating a local
  copy of it on your computer so you can view, edit, and manage the code
  offline.
\item
  \textbf{When to clone}: You clone a repository if you have permission
  to push code directly to it (e.g., you're a collaborator) or if you
  have forked it and want to work on your own copy locally.
\end{itemize}

\subsubsection{1.3 Commit}\label{commit}

\begin{itemize}
\tightlist
\item
  \textbf{What it means}: A commit is a snapshot of your changes,
  recording what changed in the code and (ideally) why.
\item
  \textbf{When to commit}: Commit often---whenever you have made a
  logical unit of change or have reached a stable state. Each commit
  should be accompanied by a clear message describing what changed.
\end{itemize}

\subsubsection{1.4 Push}\label{push}

\begin{itemize}
\tightlist
\item
  \textbf{What it means}: ``Push'' sends your local commits to a remote
  repository (on GitHub).
\item
  \textbf{When to push}: Push after you have made a series of commits
  (or at least one) that you want to share or back up on GitHub.
\end{itemize}

\subsubsection{1.5 Pull}\label{pull}

\begin{itemize}
\tightlist
\item
  \textbf{What it means}: ``Pull'' fetches any changes from the remote
  repository and merges them into your local copy.
\item
  \textbf{When to pull}: Pull before you start working and frequently
  while you work, to ensure your local copy is up to date with other
  collaborators' changes.
\end{itemize}

\subsubsection{1.6 Branch}\label{branch}

\begin{itemize}
\tightlist
\item
  \textbf{What it means}: A branch is an isolated line of development.
  The default branch is usually called \texttt{main} (or historically
  \texttt{master}).
\item
  \textbf{When to branch}: Use branches to develop features, fix bugs,
  or experiment without disturbing the main codebase.
\end{itemize}

\subsubsection{1.7 Merge \& Pull Requests
(PRs)}\label{merge-pull-requests-prs}

\begin{itemize}
\tightlist
\item
  \textbf{Merge}: Combines changes from one branch to another.
\item
  \textbf{Pull Request (PR)}: A GitHub-based mechanism to propose
  changes from one branch (or fork) to the main repository.
\end{itemize}

\subsection{2. Setting up your
environment}\label{setting-up-your-environment}

\subsubsection{2.1 Install Git}\label{install-git}

\begin{enumerate}
\def\labelenumi{\arabic{enumi}.}
\item
  If you don't already have Git installed,
  \href{https://git-scm.com/downloads}{download and install Git}.
\item
  Configure Git with your username and email (in the \textbf{terminal},
  e.g., Command Prompt, Bash, or PowerShell):

\begin{Shaded}
\begin{Highlighting}[]
\FunctionTok{git}\NormalTok{ config }\AttributeTok{{-}{-}global}\NormalTok{ user.name }\StringTok{"Your Name"}
\FunctionTok{git}\NormalTok{ config }\AttributeTok{{-}{-}global}\NormalTok{ user.email }\StringTok{"you@example.com"}
\end{Highlighting}
\end{Shaded}

  Make sure the email matches the one you use for GitHub.
\end{enumerate}

\subsubsection{2.2 RStudio and Git
Integration}\label{rstudio-and-git-integration}

\begin{enumerate}
\def\labelenumi{\arabic{enumi}.}
\tightlist
\item
  In RStudio, go to \textbf{Tools} \textgreater{} \textbf{Global
  Options} \textgreater{} \textbf{Git/SVN}.
\item
  Check that RStudio detects Git. If not, point RStudio to the path
  where Git is installed.
\end{enumerate}

\subsubsection{2.3 renv basics}\label{renv-basics}

\begin{itemize}
\tightlist
\item
  \textbf{What it is}: The \textbf{\{renv\}} R package helps you create
  isolated project environments with specific package versions. This
  ensures reproducibility across different machines and collaborators.
\end{itemize}

\begin{enumerate}
\def\labelenumi{\arabic{enumi}.}
\item
  To install \textbf{\{renv\}} in a new R session:

\begin{Shaded}
\begin{Highlighting}[]
\FunctionTok{install.packages}\NormalTok{(}\StringTok{"renv"}\NormalTok{)}
\end{Highlighting}
\end{Shaded}
\item
  (Optional) If you're creating a new project, initialize \textbf{renv}
  within that project:

\begin{Shaded}
\begin{Highlighting}[]
\NormalTok{renv}\SpecialCharTok{::}\FunctionTok{init}\NormalTok{()}
\end{Highlighting}
\end{Shaded}

  This will create a \texttt{renv} folder and a \texttt{renv.lock} file
  that tracks packages used in the project.
\item
  If you are \emph{cloning or forking an existing repository} that
  already uses \textbf{\{renv\}}, once you open the project in RStudio,
  run:

\begin{Shaded}
\begin{Highlighting}[]
\NormalTok{renv}\SpecialCharTok{::}\FunctionTok{restore}\NormalTok{()}
\end{Highlighting}
\end{Shaded}

  to install the required packages matching the versions specified in
  \texttt{renv.lock}.
\end{enumerate}

\subsection{3. Working with a repository as a
collaborator}\label{working-with-a-repository-as-a-collaborator}

If you are officially added as a collaborator on a GitHub repo (meaning
you have permission to push changes directly), follow these steps:

\subsubsection{3.1 Clone the repository}\label{clone-the-repository}

\begin{enumerate}
\def\labelenumi{\arabic{enumi}.}
\item
  Open the repository on GitHub in your web browser.\\
\item
  Click the green ``Code'' button and copy the HTTPS URL (e.g.,
  \texttt{https://github.com/username/repo.git}).
\item
  In \textbf{RStudio}, go to \textbf{File} \textgreater{} \textbf{New
  Project} \textgreater{} \textbf{Version Control} \textgreater{}
  \textbf{Git}.\\
\item
  Paste the repository URL, select a local directory to place the
  project, and click ``Create Project''.

  \begin{itemize}
  \item
    \emph{Alternatively}, clone via the \textbf{terminal}:

\begin{Shaded}
\begin{Highlighting}[]
\FunctionTok{git}\NormalTok{ clone https://github.com/username/repo.git}
\end{Highlighting}
\end{Shaded}
  \item
    Once cloned, open the \texttt{.Rproj} file in RStudio if the project
    includes one.
  \end{itemize}
\end{enumerate}

\subsubsection{3.2 Set up renv (if the repo uses
renv)}\label{set-up-renv-if-the-repo-uses-renv}

\begin{enumerate}
\def\labelenumi{\arabic{enumi}.}
\item
  In the RStudio \textbf{Console}, run:

\begin{Shaded}
\begin{Highlighting}[]
\NormalTok{renv}\SpecialCharTok{::}\FunctionTok{restore}\NormalTok{()}
\end{Highlighting}
\end{Shaded}

  This installs all the packages needed as specified by the project's
  \texttt{renv.lock} file.
\end{enumerate}

\subsubsection{3.3 Pull the latest
changes}\label{pull-the-latest-changes}

\begin{enumerate}
\def\labelenumi{\arabic{enumi}.}
\tightlist
\item
  In RStudio, click the \textbf{Git} tab (usually in the upper-right
  pane or as a separate pane).\\
\item
  Click \textbf{Pull}.

  \begin{itemize}
  \item
    \emph{Alternatively}, in the \textbf{terminal}:

\begin{Shaded}
\begin{Highlighting}[]
\FunctionTok{git}\NormalTok{ pull}
\end{Highlighting}
\end{Shaded}

    This ensures your local copy is in sync with the remote repository.
  \end{itemize}
\end{enumerate}

\subsubsection{3.4 Create a branch or work on
main}\label{create-a-branch-or-work-on-main}

\begin{itemize}
\tightlist
\item
  \textbf{Create a new branch} (recommended for new features or fixes):

  \begin{enumerate}
  \def\labelenumi{\arabic{enumi}.}
  \tightlist
  \item
    In RStudio, under the \textbf{Git} tab, click on \textbf{Branches}
    \textgreater{} \textbf{New Branch}, and give it a name.

    \begin{itemize}
    \item
      \emph{Alternatively}, from the terminal:

\begin{Shaded}
\begin{Highlighting}[]
\FunctionTok{git}\NormalTok{ checkout }\AttributeTok{{-}b}\NormalTok{ my{-}feature{-}branch}
\end{Highlighting}
\end{Shaded}
    \end{itemize}
  \end{enumerate}
\item
  \textbf{Or} work on the \texttt{main} branch (less recommended if
  you're working in a team, but sometimes used for direct small fixes).
\end{itemize}

\subsubsection{3.5 Make changes and
commit}\label{make-changes-and-commit}

\begin{enumerate}
\def\labelenumi{\arabic{enumi}.}
\tightlist
\item
  Modify your files in RStudio.\\
\item
  Check the \textbf{Git} pane in RStudio to see changed files.\\
\item
  Stage the changes by checking the boxes next to the files you want to
  commit.\\
\item
  Click \textbf{Commit}.\\
\item
  Enter a meaningful commit message.\\
\item
  Click \textbf{Commit} again to finalize.
\end{enumerate}

\subsubsection{3.6 Push to GitHub}\label{push-to-github}

\begin{enumerate}
\def\labelenumi{\arabic{enumi}.}
\tightlist
\item
  Click the \textbf{Push} button in RStudio's Git pane.

  \begin{itemize}
  \item
    \emph{Alternatively}, from the terminal:

\begin{Shaded}
\begin{Highlighting}[]
\FunctionTok{git}\NormalTok{ push origin my{-}feature{-}branch}
\end{Highlighting}
\end{Shaded}
  \end{itemize}
\item
  This will update the remote repository on GitHub with your commits.
\end{enumerate}

\subsubsection{3.7 Create a Pull Request (if your branch is ready to
merge)}\label{create-a-pull-request-if-your-branch-is-ready-to-merge}

\begin{enumerate}
\def\labelenumi{\arabic{enumi}.}
\tightlist
\item
  Go to the repository page on GitHub.\\
\item
  You'll see a banner prompting you to ``Compare \& pull request'' for
  your recently pushed branch. Click it.\\
\item
  Fill in the PR details (title, description) and submit.\\
\item
  Your collaborators can review and merge your changes.
\end{enumerate}

\subsection{4. Working with a repository as a
fork}\label{working-with-a-repository-as-a-fork}

If you are \emph{not} an official collaborator or you prefer to keep
your own copy for separate development, you can \textbf{fork} the
repository:

\subsubsection{4.1 Fork the repository on
GitHub}\label{fork-the-repository-on-github}

\begin{enumerate}
\def\labelenumi{\arabic{enumi}.}
\tightlist
\item
  Open the repository you want to fork in your web browser.\\
\item
  Click the \textbf{Fork} button (top-right corner of the page).\\
\item
  Choose your GitHub account as the destination.\\
\item
  GitHub creates a copy (fork) of the repository in your account.
\end{enumerate}

\subsubsection{4.2 Clone your fork}\label{clone-your-fork}

\begin{enumerate}
\def\labelenumi{\arabic{enumi}.}
\item
  Go to \textbf{your} forked repository (e.g.,
  \texttt{https://github.com/your-username/repo.git}).\\
\item
  Click the green ``Code'' button and copy the URL.\\
\item
  \textbf{Clone} in the same way as above (either through RStudio's
  ``File \textgreater{} New Project \textgreater{} Version Control
  \textgreater{} Git'' or via the terminal):

\begin{Shaded}
\begin{Highlighting}[]
\FunctionTok{git}\NormalTok{ clone https://github.com/your{-}username/repo.git}
\end{Highlighting}
\end{Shaded}
\end{enumerate}

\subsubsection{4.3 Set up renv (if
applicable)}\label{set-up-renv-if-applicable}

\begin{itemize}
\item
  Inside your forked repository in RStudio, run:

\begin{Shaded}
\begin{Highlighting}[]
\NormalTok{renv}\SpecialCharTok{::}\FunctionTok{restore}\NormalTok{()}
\end{Highlighting}
\end{Shaded}

  to synchronize the packages.
\end{itemize}

\subsubsection{4.4 Syncing with the original
repository}\label{syncing-with-the-original-repository}

If the original repository (often called ``upstream'') has updates, you
can pull those updates into your fork. First, set the upstream remote
link:

\begin{Shaded}
\begin{Highlighting}[]
\CommentTok{\# From inside your local forked repo}
\FunctionTok{git}\NormalTok{ remote add upstream https://github.com/original{-}owner/repo.git}
\end{Highlighting}
\end{Shaded}

Then, whenever the original repo changes, you can do:

\begin{Shaded}
\begin{Highlighting}[]
\FunctionTok{git}\NormalTok{ pull upstream main}
\end{Highlighting}
\end{Shaded}

(This fetches and merges the latest changes from the original repo into
your local fork.)

\subsubsection{4.5 Commit and push changes to your
fork}\label{commit-and-push-changes-to-your-fork}

\begin{itemize}
\item
  Commit and push works similarly:

\begin{Shaded}
\begin{Highlighting}[]
\FunctionTok{git}\NormalTok{ add .}
\FunctionTok{git}\NormalTok{ commit }\AttributeTok{{-}m} \StringTok{"Your commit message"}
\FunctionTok{git}\NormalTok{ push origin main}
\end{Highlighting}
\end{Shaded}
\item
  Or through RStudio's Git pane with \textbf{Commit} and \textbf{Push}.
\end{itemize}

\subsubsection{4.6 Open a Pull Request from your fork to the original
repo (if you want to contribute
back)}\label{open-a-pull-request-from-your-fork-to-the-original-repo-if-you-want-to-contribute-back}

\begin{enumerate}
\def\labelenumi{\arabic{enumi}.}
\tightlist
\item
  Go to your fork on GitHub.\\
\item
  Click ``Contribute'' \textgreater{} ``Open pull request'' or ``Compare
  \& pull request''.\\
\item
  Choose your fork's branch as ``head'' and the original repo's
  \texttt{main} branch as ``base''.\\
\item
  Submit your Pull Request.
\end{enumerate}

\subsection{5. Good Practices}\label{good-practices}

\begin{enumerate}
\def\labelenumi{\arabic{enumi}.}
\item
  \textbf{Pull often}: Before starting new work, make sure you have the
  latest changes from your collaborators.\\
\item
  \textbf{Meaningful commit messages}: Write short but descriptive
  messages, e.g., \emph{``Fix bug in data loading function.''}\\
\item
  \textbf{Use branches}: Keep the main branch stable. Use feature
  branches for new work.\\
\item
  \textbf{RStudio Projects}: Always open the \texttt{.Rproj} file so
  that your environment is correctly set up.\\
\item
  \textbf{renv}: Keep \texttt{renv.lock} updated if you install or
  update packages:

\begin{Shaded}
\begin{Highlighting}[]
\NormalTok{renv}\SpecialCharTok{::}\FunctionTok{snapshot}\NormalTok{()}
\end{Highlighting}
\end{Shaded}

  This ensures other collaborators know about and can install the new
  package versions.
\end{enumerate}

\subsubsection{Summary}\label{summary}

\begin{itemize}
\tightlist
\item
  \textbf{Fork} if you want your own copy and don't have collaborator
  push rights.\\
\item
  \textbf{Clone} either the main repo (if you're a collaborator) or your
  fork.\\
\item
  \textbf{Commit} local changes frequently with good messages.\\
\item
  \textbf{Push} them to GitHub to back up or share your work.\\
\item
  \textbf{Pull} updates from GitHub to stay in sync with others'
  changes.\\
\item
  \textbf{Use branches} for separate lines of development.\\
\item
  \textbf{Use renv} to manage R package dependencies and ensure
  reproducibility.
\end{itemize}

With these steps in place, you'll be well prepared to contribute to a
GitHub repository collaboratively or maintain your own forked project
with best practices for version control and environment management.

\begin{center}\rule{0.5\linewidth}{0.5pt}\end{center}

\textbf{Reference}

Text prompt: ``Please write stepwise clear instructions about how to
start working on a GitHub repo\ldots{}''

Response by ChatGPT: (2025, January 2). ChatGPT response to the prompt
``Please write stepwise clear instructions about how to start working on
a GitHub repo\ldots{}'' {[}Large language model output{]}. OpenAI.
https://chat.openai.com/




\end{document}
